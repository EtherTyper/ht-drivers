% Created 2009-04-03 Fri 12:34
\documentclass[12pt,a4paper]{article}
\usepackage[T1]{fontenc}
\usepackage{graphicx}
\usepackage{longtable}
\usepackage{hyperref}
\usepackage{ae,aecompl}
% The following is needed in order to make the code compatible
% with both latex/dvips and pdflatex.
\ifx\pdftexversion\undefined
\usepackage[dvips]{graphicx}
\else
\usepackage{graphicx}
\DeclareGraphicsRule{*}{mps}{*}{}
\fi

\title{CVORG Software Support}
\author{Emilio G. Cota $<$\href{mailto:emilio.garcia.cota@cern.ch}{emilio.garcia.cota@cern.ch}$>$ \\ CERN BE/CO/HT}
\date{03 April 2009}

\begin{document}

\maketitle


\section*{User Requirements}
\label{sec-1}

  The CVORG is a VME-based arbitrary waveform generator, featuring:
\begin{itemize}
\item Configurable sampling frequency, up to 100 Msamples/s, 14 bits per sample
\item Waveform sequencing: sets of waveforms can be configured to be played
    consecutively with no switching delay.\footnote{There is a limitation,
    though; in the case of external \emph{Event Stop} trigger configuration, the
    clock frequency of the FPGA imposes a minimum time interval between
    waveforms: 0.3us. }
\item External TTL start/stop trigger pulses with configurable polarity
    (board inputs \verb~START~, \verb~STOP~, \verb~EVSTART~, \verb~EVSTOP~)
\item Two independent generators per card; each generator has memory for
    1Msamples
\item Configurable Unipolar (0/10V) or Bipolar (-10V/+10V) output
\end{itemize}
  Support from the CO/HT section comprises a Linux/LynxOS driver and a
  user-space library to use the module. All our drivers implement
  synchronisation mechanisms to deal with concurrent access to the module,
  which is especially important when there is a problem with an operating
  module. We also develop test routines that help debugging the module and
  software; these programs serve as well as diagnose tools once a module has
  been delivered.

\section*{Library API}
\label{sec-2}

\subsection*{\emph{Note, Note, Note!}}
\label{sec-2.1}

   \emph{The following API description mustn't be understood as a contract;}
   \emph{it is just meant to give a rough idea of the end result to make sure that}
   \emph{we are suiting our users' needs. It should certainly evolve as the}
   \emph{development progresses.}

\subsection*{Why a library?}
\label{sec-2.2}

  By providing access to the hardware via a library (instead of making direct
  calls to the driver) we can offer higher abstraction of the module taking
  care of tedious tasks (such as endianness issues) along the way, making
  the module easy to use and integrate with other higher-level libraries.

\subsection*{Selecting a module}
\label{sec-2.3}

  The first thing to do is to select and configure a module to work on;
  this is done by filling \verb~struct cvorg~ and passing it to \verb~cvorg_init()~.
  \verb~struct cvorg~ has the following fields:
\begin{itemize}
\item \verb~int mod_nr~: hardware module (card) to work on.
\item \verb~int gen_nr~: generator number (1 or 2).
\item \verb~int trig_pl~: trigger polarity (\verb~CVORG_{RISING,FALLING}_EDGE~.)
\item \verb~int func_pl~: waveform polarity (\verb~CVORG_{BI,UNI}POLAR~.)
\item \verb~int sequence~: 0 for infinite loop; n > 0 for playing a sequence n times
    (more on sequences later.)
\item \verb~int sampclk~: sampling clock:

\begin{itemize}
\item \verb~CVORG_CLK_DEFAULT~: internal clock, 100MS/s
\item \verb~CVORG_CLK_EXTERNAL~: external clock
\item \verb~CVORG_CLK_GENERATE~: generate a certain sampling frequency
\end{itemize}

\item \verb~float clockfreq~: desired value for sampling frequency (in the case
    \verb~sampclk~ equals \verb~CVORG_CLK_GENERATE~). Note that this value cannot be
    bigger than 100MS/s. Also note that a maximum precision for this parameter
    is yet to be decided.
\end{itemize}
  A module always has the output set to 0V unless it is playing a
  function---this is valid for both polarity modes.

\subsection*{Generating a waveform}
\label{sec-2.4}

   Once the module is configured, the user can request the generation of a
   waveform. For this purpose a \verb~struct cvorg_wave~ has to be filled:
\begin{itemize}
\item \verb~void *form~: address of the array which contains the set of points that
     define one period of the desired waveform. A point is an unsigned 16-bit
     value (type \verb~uint16_t~ in C99 standard). Note that the amplitude at the
     output depends as well on the configured polarity: in the unipolar case,
     0xFFFF will output +10V and 0x0000 will output 0V, whereas in bipolar
     mode 0xFFFF translates into +10V and 0x0000 corresponds to -10V.
     Also note that a set of points must have an even number of members;
     otherwise the last point will be discarded.\footnote{This limitation is likely
     to go away in the final version; as of this writing we have a proposed
     (though untested) solution. }
\item \verb~size_t size~: size in bytes of the array pointed by \verb~form~.
\item \verb~int recurse~: 0 for playing in an endless loop; n > 0 for playing
     this waveform n times.
\item \verb~struct cvorg_wave *next~: pointer to the next waveform in a sequence;
     NULL to end a sequence.
\end{itemize}
\subsubsection*{More on Sequences}
\label{sec-2.4.1}

    A sequence is a set of consecutive waveforms. A sequence is started by
    the \verb~START~ trigger and stopped by the \verb~STOP~ trigger. A sequence can
    be played \emph{p} number of times or infinitely (by setting \verb~sequence~ on
    \verb~struct cvorg~ accordingly). In both cases the \verb~STOP~ trigger will stop
    the sequence.

    A waveform can be played \emph{n} times or infinitely, depending on the value
    of \verb~recurse~ on \verb~struct cvorg_wave~. In both cases the \verb~EVSTOP~ trigger
    will stop the waveform, and the sequence will move on to playing the
    following waveform (if any).
    Describing a couple of examples might help to understand these concepts:
\begin{itemize}
\item Case 1. Play sequence \{A,B,C\} \emph{p} times, where \{A,B,C\} are waveforms
      with \verb~recurse~ set to $n_a$, $n_b$, $n_c$ respectively.

\begin{enumerate}
\item \verb~START~ starts the sequence
\item A is played $n_a$ times
\item B is played $n_b$ times
\item C is played $n_c$ times
\item Check how many times the sequence has been played; go to (2)
        if it hasn't been played \emph{p} times yet; end otherwise.
\end{enumerate}

\end{itemize}
    Note that anytime \verb~EVSTOP~ or \verb~STOP~ can be triggered anytime; they will
    do what they are expected to do irrespective of the state of the recursion.
\begin{itemize}
\item Case 2. Play sequence \{A,B\} forever, where \{A,B\} are waveforms with
      \verb~recurse~ set to 0.

\begin{enumerate}
\item \verb~START~ starts the sequence
\item A is played until \verb~EVSTOP~ triggers
\item B is played until \verb~EVSTOP~ triggers
\item Go to (2)
\end{enumerate}

\end{itemize}
    In this case the loop will be executed until \verb~STOP~ triggers.




\end{document}
