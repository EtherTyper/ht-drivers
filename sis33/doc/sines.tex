\documentclass{article}
\usepackage[T1]{fontenc}
\usepackage{times,mathptmx}
\usepackage{amsmath}

\title{A fast way to check whether a sine wave is actually sinusoidal}
\author{Juan David Gonz\'{a}lez Cobas}

\begin{document}

\maketitle
Let us assume a true sine wave $f(t) = A \sin(\omega t + \phi)$
of known frequency $\omega = 2\pi f = 2\pi/T$ is sampled at
$N_s$ samples-per-second rate, which amounts to a sampling period of
$T_s = 1/N_s$.
Thus, a complete sine period contains a number of samples
\begin{equation}
	N = T N_s = \frac{2\pi}{\omega} \, N_s =
	 \frac{2\pi}{\omega T_s}
\end{equation}
which is not necessarily an integer. To make more precise computations,
and getting a sampling as coherent as possible,
let us take a number~$p$  of sine periods
such that $pN$, the total amount of samples taken into account,
be as close to an integer as possible. This is not needed if the
sampling is coherent, i.e., the sampling period divides exactly the sine
wave period.

Now, let us suppose we got a vector of $pN$ samples of the (apparently)
sinusoidal wave~$f$. If it is actually sinusoidal, it must happen that
\begin{equation}
	\frac{1}{pN} \sum_0^{pN-1} f_k^2 = A^2/2
\end{equation}
(why? A hint: $\sin^2 + \cos^2 = 1$). This is nothing more than a
restatement of the fact that the RMS crest factor for sine waves
is $C_\text{rms} = 1/\sqrt 2$.

Let us now compute the normalized Fourier coefficients
\begin{align}
	C &= \frac{\sqrt 2}{pN} \sum_0^{pN-1} f_k \,\cos \frac{2\pi k}{N} \\
	S &= \frac{\sqrt 2}{pN} \sum_0^{pN-1} f_k \,\sin \frac{2\pi k}{N}
\end{align}
(the $\sqrt 2$ factors are precisely the reciprocal of $C_\text{rms}$
that normalize the vector of $\cos$ and $\sin$ we are correlating to).
Then, our set of samples corresponds exactly to a true sine wave if and
only if
\begin{equation}
	C^2 + S^2 = A^2/2
\end{equation}
\end{document}

